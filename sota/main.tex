\documentclass[11pt, a4paper]{article}
\usepackage[T1]{fontenc}
\usepackage[utf8]{inputenc}
\usepackage[english]{babel}
\usepackage{fancyhdr}
\usepackage{tabularx}
\usepackage{geometry}
\usepackage{rotating}
\usepackage{minitoc}
\usepackage{verbatim}
\usepackage{pdfpages}
\usepackage{float}
\usepackage{listings}
\usepackage{graphics}
\usepackage{listings}
\usepackage{geometry}
\usepackage{mathtools}
\usepackage{amsmath}
\usepackage{amssymb}
\usepackage{newtxtext,newtxmath}
\usepackage{siunitx}
\usepackage{float}
\usepackage[inkscapeformat=png]{svg}
\usepackage[backend=bibtex, style=ieee]{biblatex}
\usepackage{csquotes}
\usepackage{hyperref}

\geometry{
  a4paper,
  total={170mm,257mm},
  left=20mm,
  top=20mm,
}

\hypersetup{
  colorlinks=false,
  %linkcolor=blue,
  filecolor=magenta,
  urlcolor=cyan,
  pdftitle={SOTA},
  pdfauthor={Nicolas Jeanmenne}
  pdfpagemode=FullScreen,
}

\date{}
\title{State-of-the-art}
\author{Nicolas Jeanmenne}

\bibliography{biblio.bib}

\begin{document}

%%%% FOR ALL THE DOCUMENT
\parindent0pt
\newcommand{\HRule}{\rule{\linewidth}{0.5mm}}

%%%% TITLE PAGE
\begin{titlepage}
  \begin{figure}[!t]
    \centering
    \includegraphics[width=.4\textwidth]{images/ucl_logo.jpg}
  \end{figure}
  \center
  \vspace{1.35em}
  \textsc{\LARGE Direct device access from the SmartNIC towards datacenter disaggregation}\\[0.5cm]
  \textsc{\large} \\[0.5cm]
  \HRule \\[0.5cm]
  { \huge \bfseries State-of-the-art}\\[0.3cm]
  \HRule \\[1.5cm]
  \begin{minipage}[t]{.5\textwidth}
    \large{
      \vspace{0.2cm}
      \begin{center}
        Nicolas \textsc{Jeanmenne} - 48741900\\
        \vspace{0.4cm}
      \end{center}
      \vspace{5cm}

    }
  \end{minipage}%
  \vspace*{3cm}

  \begin{figure}[!h]
    \centering
    \includegraphics[width=.5\textwidth]{images/logepl.jpg}
  \end{figure}
  \vspace{0.5cm}

  {\large 2025-2026}
\end{titlepage}

\newpage

\pagestyle{fancy}
\renewcommand\headrulewidth{1pt}
\fancyhead[L]{Direct device access from the SmartNIC towards datacenter disaggregation}
\fancyhead[R]{State-of-the-art draft}
\setlength{\headheight}{14.5pt}
\newpage

\newcommand{\authcite}[1]{\citeauthor{#1} \cite{#1}}

%%
%% The abstract is a short summary of the work to be presented in the
%% article.
\begin{abstract}
  TODO: complete abstract

  \textit{This paper an is draft aiming at analyzing state-of-the-art solution for DDA from the smartNIC towards datacenter disaggregation
  and is not made to be submitted anywhere}
\end{abstract}

\maketitle

\section{Introduction}

For decades, servers in datacenters were designed under a monolithic model, meaning that all hardware resources are physically
attached to the same frame. This model is now a bottleneck for performance, network throughput and to efficiently consume resources.
In order to solve the monolithic challenge, a new architectural design has been proposed : disaggregation.

\section{Definitions}

\subsection{Disaggregation}

Disaggregation is a new model where hardware resources such as computing power, memory, accelerators, storage, ... are divided
into pools (or nodes) interconnected into each other via a high-speed network. This allows resources to be allocated on demand,
with better utilization and availability. There is mainly two levels of disaggregation.

Partial disaggregation separates storage components from CPU and memory which remain integrated together. This level of
disaggregation has already been widely implemented since early 2020s.

Fully disaggregation regroups same type of resource into clusters which are interconnected over a network to allow
communication and data exchange between them. \cite{lin_disaggregated_2020}

\subsection{RDMA}

Remote Direct Memory Access (RDMA) is a protocol that allows to access memory from one computer into another computer without
involving operating system \cite{recio_remote_2007}. This protocol offers high throughput and low latency, and will be referred from many articles
covering this topic.

\subsection{Zero-copy}

Zero-copy enable the possibility to transfer data from one device to another without requiring CPU to copy the data. This
avoids redundant copies, reducing CPU usage.

\subsection{NVMe-oF}

NVMe over-Fabric (NVMe-oF) is an extension of NVMe command set. It allows to send these commands over networked fabrics
\cite{noauthor_nvme_2021, noauthor_what_nodate, guz_nvme-over-fabrics_2017}



\section{Problems and challenges}

Problems can be divided into two boxes by their root cause :

\begin{itemize}
  \item Problems due to traditional monolithic architectural server. This type of problem is the core reason of why datacenters
    are moving from monolithic architecture from disaggregated one.
  \item New challenges brought by disaggregation. Disaggregated computing is a relatively new design in system and datacenter
    who breaks many of assumption previously made in the research. Researchers and datacenter operators need to find
    creative solutions to outcome these challenges and enjoy the benefits of disaggregation.
\end{itemize}

\subsection{Monolithic challenges}

The main drawback with monolithic architecture (MA) is \textbf{resource stranding}. When a server exhausts one of its resource,
for example memory, all other resources cannot be utilized and result in a waste of resources.

Another non-negligible issue concerning MA is failure tolerance and availability. When a piece of hardware fails this may cause
the entire server to fails and become unavailable.

Finally, replacing or upgrading hardware in MA occurs maintenance that can be costly.

\subsection{Disaggregation challenges}

One of the main challenge in disaggregated architecture (DA) is that network bandwidth becomes a central point of communication between
clusters; CPU-storage communication needs to be done in a few milliseconds while CPU-memory needs to be done in a few nanoseconds, usually
100ns \cite{lin_disaggregated_2020}.

Existing network stacks face significant trade-offs in terms of flexibility and performance. For example, RDMA and RDMA
over Convergent Ethernet (RoCE) are high-performance protocols, but they lack flexibility because they are tightly coupled
to specific hardware, making them difficult to adapt. In addition to limited flexibility, RDMA suffers from issues like
head-of-line blocking, congestion, and vendor lock-in, due to hardware requirements that aren't met by commodity equipment.
On the other hand, network stacks like Linux TCP offer greater flexibility but at the cost of poor performance and
inefficient CPU utilization \cite{skiadopoulos_high-throughput_2024}.

\section{State-of-the-art designs and solutions}

\subsection{Zero-copy data path}

\authcite{skiadopoulos_high-throughput_2024} designed a new prototype called \textit{ZeroNIC} which can zero-copy data between
NIC and accelerators. Its main advantage is that is it agnostic of the accelerator type and canrun on CPU, GPU, FPGA, ...
NIC hardware split header and payload, transferring the latter directly to the receiver applications buffers without intermediates
copies.

Concerning NVMe storage, \authcite{sun_scalio_2025} focused on NVMe-oF target offloading which extends zero-copy mechanism,
transferring data to the DPU's host channel adapter (HCA) via peer-to-peer PCI communication (P2PDMA) and thus creating a new
kind of zero-copy data path.

\section{Objectives}

This paper will focus on fully disaggregated datacenter, especially on interaction between NVMe storage pools and SmartNIC.
One of our objectives is to allow SmartNICs to communicate directly with the storage without involving any compute pool.
%TODO:  give a more precise definition

\section{Conclusion}
TODO : write conclusion

\printbibliography

\end{document}